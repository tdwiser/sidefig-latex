\documentclass{article}

\usepackage{sidefig}

\usepackage{url}

\begin{document}
	\title{\texttt{sidefig} Package Example}
	\author{Tim Wiser}
	\date{\today}
	
	\maketitle

	\newsavebox\samplebox
	\savebox{\samplebox}{\fbox{\vbox{\vskip 4ex \hbox{\hskip 8ex \it figure \hskip 8ex} \vskip 4ex}}}
	
	\sidefig{L}{\usebox{\samplebox}}
	Wrapping text around a figure (or, more generally, any box-shaped region) is a tricky subject in \LaTeX.
	Plain \TeX\ has a powerful command, \verb|\parshape|, which directs \TeX\ to flow a paragraph within a defined region, with the geometry defined line-by-line.
	
	Various packages (e.g.~\texttt{wrapfig}) exist to facilitate this process for \LaTeX\ users.
	However, \verb|\parshape| (and another powerful command, \verb|\everypar|, which allows wrapping across multiple paragraphs) are overwritten by \LaTeX\ \verb|list| environments, including \verb|itemize| and \verb|enumerate|.
	This causes typical approaches to fail when used near \verb|list|s.
	
	\texttt{sidefig} is a package which attempts to allow wrapping text around rectangular regions even in the presence of \texttt{list}s.


	Here are some features of \texttt{sidefig}, which you can see demonstrated below:
	\begin{itemize}

		\sidefig{R}{\usebox{\samplebox}}
		\item Text wrapping continues to work even if you start a list.
		For example, this line will break a little bit earlier to make room for the figure at right.
		
		\item Sublists are also supported:
		

		\begin{enumerate}
			
			\item like so.

			\sidefig{L}{\usebox{\samplebox}}
			\item If you add a \verb|\sidefig| in a list, the package tries hard to make things look right even across multiple levels.
		\end{enumerate}
		
		\item Now we're back in the outer list and the indent is hopefully right.
			
		\sidefig[Fig.~1: A caption describing the figure above.]{R}{\usebox{\samplebox}}
		\item Captions are also possible using an optional parameter to \verb|\sidefig|.
		
		\item Usage:
		
		\verb|\sidefig[|\emph{(caption)}\verb|]{L|\emph{ or }\verb|R}{|\emph{(content)}\verb|}| right before any paragraph or \verb|\item|.
	\end{itemize}
	
	% \pagebreak %%% this is necessary to appease the coming sidefig, which would otherwise be typeset across a page boundary
	
	\section*{Additional Features}

	\renewcommand{\sidecapstyle}[1]{\uppercase{#1}}
	\sidefig[Caption with redefined style]{R}{\usebox{\samplebox}}
	The style of the caption can be controlled by redefining \verb|\sidecapstyle|.
	The default is
	\begin{quote}
		\verb|\def\sidecapstyle{\centering\small\it}|.
	\end{quote}
	\verb|\sidecapstyle| can also take an option, if you like, as in
	\begin{quote}
		\verb|\renewcommand{\sidecapstyle}[1]{\uppercase{#1}}|.
	\end{quote}
	
	
	The spacing between the content and the caption can be controlled with
	\begin{quote}
		\verb|\setlength{\sidecapskip}{|\emph{(length)}\verb|}|,
	\end{quote}
	default \verb|2pt|.
	
	If you don't want to be next to the \verb|\sidefig| any more, use \verb|\MoveBelowBox|.
	This command is called automatically if you add a \verb|\sidefig| that would otherwise overlap a previous one.
	
	\subsection*{Known Issues}
	
	\begin{itemize}
		\item Various issues around page breaks, since \TeX\ works ahead while building pages, deciding the breaks later.
		Current behavior is to raise a warning if the sidefig would go below the ostensible end of the page.
		Version rc0.3 improves the behavior somewhat, and the package will attempt to recover across a page break.
		Sometimes this leads to some bad behavior if there are multiple paragraphs next to a \verb|\sidefig|; the first and second will appear on different pages, but the second will still have a cutout for the figure which fell on the previous page.

		\item There should NOT be a blank line between \verb|\sidefig| and the next paragraph or (especially) \verb|\item|.
		In order to make the positioning right within lists, \verb|sidefig| has to look ahead at the next command, which does not work across a blank line.
		
	\end{itemize}
	
	\subsection*{Acknowledgements}
	
	This code is heavily based off of the Plain \TeX\ package \verb|insbox| by Micha\l{} Gulczy\'nski, available at \url{https://ctan.org/pkg/insbox}.
	
	Not all features of \verb|insbox| are available due to poor interactions with \verb|list|s.
	Many things have been wrapped-around or otherwise rearranged to accommodate \verb|list|s, but the core functionality is due to \verb|insbox|.
	
\end{document}
